\documentclass[../main.tex]{subfiles}   
   \begin{document}
   \section{Venerdi 5/7}
   \subsection{Timetable}
   \begin{itemize}
        \item 7:30 Sveglia
        \item 8:00 - 10:00 Ginnastica, colazione e ispezione
        \item 10:00 - 12:30 Giuochi d'acqua + Doccia
        \item 12:30 - 15:00 Pranzo, servizi, TL
        \item 15:00 - 16:30 Bottega
        \item 16:30 - 17:30 Merenda + Catechesi
        \item 17:30 - 19:30 \textbf{Gioco}
        \item 19:30 - 21:00 Cena
        \item 21:00 - 22:30 Veglia alle stelle
        \item 22:30 Nanna!
    \end{itemize}

   \subsection{10:00 - 12:30 Giuochi d'acqua}
       \subsubsection{Lancio}
       La scena inizia con due pellegrini, che passando vicino alla tana della Vecchia Shelob, ne narrano le
storia:
“Shelob era arrivata prima di Sauron e non serviva altri che se stessa, bevendo avidamente il sangue
di Elfi e Uomini grassa e gonfia per via dell'interminabile rimuginare i suoi banchetti, tessendo
ragnatele d'ombra; ogni essere vivente era il suo cibo, e il suo vomito era oscurità.”
Dopo qualche secondo arrivano Frodo e Sam , condotti davanti alla tana da Gollum, deciso di poter
prendere l'Unico Anello dopo che lei li avesse mangiati, poiché lei non aveva interesse ad esso. I
due Hobbit chiedono l'aiuto delle mistiglie, per superare i cunicoli e prove che conducono da quella
orribile creatura, e la loro forza e furbizia per sconfiggerla.
Chi? Frodo, Sam, se può Gollum + due pellegrini random

        \subsubsection{Gioco}
        Percorso in tre parti. A tempo, finchè tutta la mistiglia non supera la sezione non si può passare a quella successiva. \\
\\
Le frecce degli orchi: prato da superare di corsa. Due o tre di noi ai lati che lanciano gavettoni (frecce). Chi viene colpito ricomincia da capo.\\
\\
Il grande Uruk hai/la salita di cirith ungol: la base è la stessa, un telone cerato 29. Se si trova una salita si può fare una salita saponata con noi che dall’alto buttiamo acqua sapone e gavettoni. Se si sta in piano si deve affrontare due uruk hai, che stile bulldog tenteranno di solevarti da terra, il tutto sul telo saponato e scivoloso.\\ 
\\
La vecchia shelob: bisogna uccidere il grande ragno, con una spada: un filo con gavettoni appesi, con uno stuzzicadenti in bocca vanno fatti scoppiare. 
        \subsubsection{Materiali}
        \begin{itemize}
            \item gavettoni
            \item acqua
            \item sapone
            \item un telo cerato 29
            \item una corda
        \end{itemize}
        \subsubsection{Note}
        \subsubsection{Spiega}

    \subsection{15:00 - 16:30 Bottega}
   cfr \textbf{Botteghe}
    
    \subsection{16:30 - 17:30 Catechesi}   
    \subsection{17:30 - 19:30 Gioco Quicker (palla di Brea)}
        \subsubsection{Lancio}
        \subsubsection{Gioco}
         Quicker (palla di Brea)
        \subsubsection{Materiali}
        \subsubsection{Note}
        \subsubsection{Spiega}

    \subsection{21:00 - 22:30 Veglia alle stelle}
        \subsubsection{Lancio}
        Gandalf racconta 4 storie di 4 diverse costellazioni.\\
        Alla fine delle storie entrano nella scenetta i 4 protagonisti che iniziano a litigare. Entra un quinto personaggio, una costellazione nuova che non ha nessuna storia.\\
        \subsubsection{Gioco}
        I fratellini, divisi per mistiglie, dovranno scrivere una storia per il quinto personaggio. La storia sarà scritta a mo'di fisarmonica (come abbiamo scelto il tema del campo).
        Alla fine verranno a crearsi n fisarmoniche per mistiglia (una per ogni fratellino). La mistiglia ne sceglierà una da presentare e leggere al branco.\\
        Tornati in cerchio si condivideranno le stelle
        \subsubsection{Materiali}
        \begin{itemize}
            \item Quaderni di caccia e penne
        \end{itemize}
        \subsubsection{Note}
        Terminato il gioco i fratellini e le sorelline si stenderanno sul telo 29 a guardare le costellazioni di cui hanno appena sentito raccontare.
        \subsubsection{Spiega}
   \end{document}