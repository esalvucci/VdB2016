\documentclass[../main.tex]{subfiles}   
   \begin{document}
   \section{Martedi 2/7}
   \subsection{Timetable}
   \begin{itemize}
        \item 7:30 Sveglia
        \item 8:00 - 10:00 Ginnastica, colazione e ispezione
        \item 10:00 - 12:30 Partenza + Racconto 1 + GiocoGiungla 1
        \item 12:30 - 15:00 Pranzo \textbf{al sacco}
        \item 15:00 - 16:30 Racconto 2
        \item 16:30 - 17:30 \textbf{Ankus}!!!
        \item 17:30 - 19:30 Doccia
        \item 19:30 - 21:00 Cena
        \item 21:00 - 22:30 Caccia d'atmosfera
        \item 22:30 Nanna!
    \end{itemize}

    \subsection{8:00 - 10:00 Ginnastica, colazione e ispezione}
        \subsubsection{Note}
        Durante la colazione vogliamo fare un lancio della caccia? tipo la compagnia dell'anello che si ritrova per partire \\ 
        Ci sarebbe l'effetto sorpresa perchè gli viene detto durante la colazione (se dopo l'ispezione dicessimo rimanete in pelliccia sarebbe ovvia la caccia) d'altra parte hanno tempo per prepararsi all'idea di dover partire.
    \subsection{10:00 - 12:30 Racconto 1 + Gioco Giungla 1}
       \subsubsection{Lancio}
        Racconta: Wontolla
        \subsubsection{Gioco}
        2 squadre: da una parte lettere e dall'altra numeri come in sultano (chi deve prendere il tesoro e chi deve impedire che il tesoro venga preso). Al centro il tesoro (monete gialle, che devono essere difese): attaccante (colui che deve prendere il tesoro) e difensore si sfidano allo scalpo (1 contro 1)\\
       Ad un certo punto del gioco (a sorpresa) un vecchio lupo urlerà “Il cobra bianco è morto!”. A questo punto sarà un “tutti contro tutti”: I fratellini dovranno prendere l'ankus e il più veloce che riuscirà a portarlo all'uscita della caverna (punto x sul campo) partirà con l'ankus nel gioco sucessivo (ma temporaneamente gli diremo solo che ha vinto quel gioco)

        \subsubsection{Materiali}
            \begin{itemize}
                \item n / 2 scalpi (uno per ogni componente della squadra di attacco)
                \item Ankus
                \item Monete per rappresentare il tesoro (al centro)
            \end{itemize}
        \subsubsection{Note}
        \subsubsection{Spiega}

    \subsection{15:00 - 16:30 Racconto C.S.I.}
       \subsubsection{Lancio}
       Racconta: Akela, "elementare Watson"!
       \subsubsection{Gioco}
       Il fratellino che ha vinto il gioco precedente avrà 1 minuto di vantaggio rispetto al resto del branco.\\
       Il fratellino dovrà ripercorrere la strada fatta la mattina al contrario con l'ankus in mano per tornare alla casa. Ogni 20 passi dovrà fischiare in modo che si sappia +/- la sua distanza e dopo aver fischiato dovrà lasciare per terra delle stelle filanti (la sua traccia).\\
       Nel frattempo gli altri fratellini rincorreranno il primo e se un altro fratellino lo raggiunge allora il primo corridore gli cederà fischietto, ankus e stelle filanti.
       Chi arriverà alla casa con l'ankus potrà decidere se tenerselo o donarlo al branco.
       
    \subsubsection{Materiali}
       \begin{itemize}
            \item Ankus
            \item Stelle filanti
            \item Fischietto
            \item Walkie-talkie
       \end{itemize}
       \subsubsection{Note}
       Valutiamo se cambiare il gioco visto che non centra con il racconto.
        \subsubsection{Spiega}
    
    \subsection{21:00 - 22:30 Caccia d'atmosfera}
        \subsubsection{Lancio}
        Quattro personaggi: Sant’antonio, saan francesco, santa rita(baghi) e papa giovanni paolo II
    
        \subsubsection{Gioco}

        \subsubsection{Materiali}
        \subsubsection{Note}
        \subsubsection{Spiega}
   \end{document}