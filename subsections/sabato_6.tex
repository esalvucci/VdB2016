\documentclass[../main.tex]{subfiles}   
   \begin{document}
   \section{Sabato 6/7}
   \subsection{Timetable}
   \begin{itemize}
        \item 7:30 Sveglia
        \item 8:00 - 10:00 Ginnastica, colazione e ispezione
        \item 10:00 - 12:30 Bottega
        \item 12:30 - 15:00 Pranzo, servizi, TL
        \item 15:00 - 16:30 Caccia al tesoro
        \item 16:30 - 17:30 Merenda + Catechesi
        \item 17:30 - 19:30 \textbf{Gioco}
        \item 19:30 - 21:00 Cena
        \item 21:00 - 22:30 Fiesta
        \item 22:30 Nanna!
    \end{itemize}

   \subsection{10:00 - 12:30 Bottega}
   cfr \textbf{Botteghe}

    \subsection{15:00 - 16:30 Caccia al tesoro}
        \subsubsection{Lancio}
        Lancio: Aragorn (Wantolla) racconta la leggenda della spada, l’unica spada in grado di uccidere Sauron, che è stata distrutta. Inviterà tutte le mistiglie alla ricerca dei pezzi della spada per ricomporla e per salvare la Terra di Mezzo. 
        \subsubsection{Gioco}
        Aragorn dà il primo messaggio alle mistiglie.
Stili dei messaggi: Codice morse; indovinello; criptato (alfabeto elfico)
Nell’ultimo indizio c’è scritto il posto dove si trova il vecchio lupo con il pezzo della spada e viene indicata la tecnica con la quale la mistiglia deve arrivarci (serpentone dove tutti sono bendati tranne l’ultimo della fila che dà le indicazioni a tutti).  
        \subsubsection{Materiali}
        \begin{itemize}
            \item Codice morse
            \item alfabeto elfico
            \item Pezzi spada
            \item Bandiere per morse
        \end{itemize}
        \subsubsection{Note}
        \subsubsection{Spiega}       
    \subsection{16:30 - 17:30 Catechesi}   
    \subsection{17:30 - 19:30 Gioco}
        \subsubsection{Lancio}
        \subsubsection{Gioco}
        
        \subsubsection{Materiali}
        \subsubsection{Note}
        \subsubsection{Spiega}    
        
    \subsection{21:00 - 22:30 Fiesta}
        \subsubsection{Lancio}
        \subsubsection{Gioco}
        ENT ROSSO (alce rossa)\\
        Le mistiglie sono oramai nella piana di mordor, che pullula di orchi!! Dovranno riuscire a superare indenni il lungo viaggio che li aspetta, ma il pan di via è poco!!\\
\\
Ogni squadra ha del pan di via nella propria base, lo scopo è procacciarsene dell’altro, rubandolo alle altre mistiglie. Chi viene beccato dagli altri ffss deve tornare in un punto a cambiare l’alce rossa prima di ripartire. \\
Difficoltà in più: ci saranno degli orchi (noi) che girano per la piana e se ti leggono la scritta in testa ti portano in prigione. Per uscire di prigione un altro componente deve entrare senza farsi leggere in testa e portare via l’amico. L’uscita è libera. (la variante era che per uscire o si sfida un orco a scalpo o si risponde agli indovinelli, ma mi pare tropp complesso.) \\
\\
Alla fine vince chi ha più cibo con se. Se qualche membro della squadra è in prigione è un malus. \\
\\Dopo viene dichiarata la mistiglia vincitrice del campo, che avrà l grande onore di buttare l’anello nel monte fato. \\
\\
Il vulcano farà fumo e tutti festeggiano la sconfitta del male. 
        \subsubsection{Materiali}
        \begin{itemize}
            \item alci rosse
            \item foglietti con su pane/oggetti che lo simboleggino
            \item vulcano (mang/ferao???)
            \item hiaccio secco e acqua calda per il fumo
        \end{itemize}
        \subsubsection{Note}
        Dobbiamo decretare i vincitori del campo durante la staff serale del 5?!?
        \subsubsection{Spiega}
   \end{document}