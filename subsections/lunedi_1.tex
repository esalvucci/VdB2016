\documentclass[../main.tex]{subfiles}   
   \begin{document}
   \section{Lunedi 1/7}
   \subsection{Timetable}
   \begin{itemize}
        \item 7:30 Sveglia + Auguri Raksha!!!
        \item 8:00 - 10:00 Ginnastica, colazione e ispezione
        \item 10:00 - 12:30 Rifugi di mistigliae, urli e stemmi
        \item 12:30 - 15:00 Pranzo, servizi, TL
        \item 15:00 - 16:30 Gioco compagnia dell'anello
        \item 16:30 - 17:30 Merenda + Catechesi
        \item 17:30 - 19:30 Calcio gaelico
        \item 19:30 - 21:00 Cena
        \item 21:00 - 22:30 Fuoco
        \item 22:30 Nanna!
    \end{itemize}

   \subsection{10:00 - 12:30 Rifugi mistiglie/Urlo/Stemma}
       \subsubsection{Lancio}
       Ogni famiglia che si rispetti ha un luogo al quale è affezionata, un luogo dove raccontarsi storie alla
luce delle stelle, dove ci si può davvero sentire a casa.
Ognuno dei quattro personaggio racconterà dei propri ricordi e dei propri luoghi di provenienza,
invitando le quattro mistiglie a crearne uno proprio con le cose che li circondano.
Chi? Sam-Gimli-Galadriel-Aragorn

        \subsubsection{Gioco}
        Costruzione rifugi mistiglie/Urlo/Stemma
        \subsubsection{Materiali}
        \begin{itemize}
            \item Cordini
            \item Teli
            \item Cianfrusaglie
        \end{itemize}
        \subsubsection{Note}
        Durante questa attività i fratellini e sorelline dovranno anche pensare di mistiglia ad un gioco e un canto da proporre durante il fuoco serale
        \subsubsection{Spiega}
        
    \subsection{15:00 - 16:30 Gioco compagnia dell'anello}
       \subsubsection{Lancio}
       Il Consiglio di Elrond si riunisce a Gran Burrone, dove si decide di distruggere l'Anello portandolo a Mordor e gettandolo nel Monte Fato, dove è stato forgiato
       Frodo si offre volontario per compiere la missione. Da solo però non può farcela (ricordiamoci che è poi sempre solo un hobbit) e quindi ha bisogno di qualcuno che lo aiuti.
       \subsubsection{Gioco}
       Ci sono quattro basi, una per mistiglia. Ogni mistiglia ha un colore (che è anche quello della maglia e dello stemma). Ogni componente della mistiglia ha un pezzo di lana del colore della mistiglia. I ff/ss si sfidano tra di loro a scalpo. Chi vince prende lo scalpo e accompagna il perdente dal suo vecchio lupo di riferimento. Consegna lo scalpo al vecchio lupo ed è libero di andare a sfidare altri ff/ss. Il perdente deve rispondere ad una domanda, da parte del vecchio lupo, sulla mistiglia rivale. Se risponde bene rimane nella sua mistiglia di appartenenza conservando il suo pezzo di lana. Se risponde male diventerà parte della mistiglia avversaria, consegnerà il suo pezzettino di lana e ne avrà in cambio uno del colore della sua nuova mistiglia
       \subsubsection{Materiali}
       \begin{itemize}
           \item Scalpi (Tutti)
           \item 4 gomitoli di colore diverso tra loro (Kaa)
           \item domande (?)
       \end{itemize}
       \subsubsection{Note}
       \subsubsection{Spiega}

    \subsection{16:30 - 17:30 Catechesi}   
    \subsection{17:30 - 19:30 Calcio gaelico}
        \subsubsection{Lancio}
        \subsubsection{Gioco}
        Il campo da gioco è diviso in due: in una metà campo si gioca a rugby lupetto e nell'altra a calcio
        \subsubsection{Materiali}
        \subsubsection{Note}
        \subsubsection{Spiega}
        
    \subsection{21:00 - 22:30 Fuoco}
        \subsubsection{Lancio}
        Finito il kamaludu il primo personaggio si presenta e lancia il primo gioco.

        I personaggi racconteranno la loro storia a coppie (Ogni personaggio pensa alla propria presentazione).
        \subsubsection{Gioco}
        Tema del fuoco: fare comunità/stare assieme
        
        Ogni personaggio dopo essersi presentato chiederà ad una mistiglia di insegnargli un ban e un canto in modo alternato.

        Cioè:
        \begin{itemize}
            \item Coppia di personaggi 1
            \item Mistiglia 1 con un ban
            \item Coppia di personaggi  2
            \item Mistiglia 2 con un canto.
            \item E così via per il 3 e 4
        \end{itemize}

        Contributi dei personaggi si alternano a contributi delle mistiglie.
        Le mistiglie, durante l'attività “Rifugio/Urlo/Stemma”, pensano ad un gioco e ad un canto da proporre. Le mistiglie scriveranno il loro gioco e canto in un biglietto messo dentro una busta che daranno ai vecchi lupi durante la cena. In questo modo sappiamo quali ban e canti faremo e se ci sono sovrapposizioni ci prepariamo delle alternative

        Faremo i canti e i giochi delle mistiglie + 1 canto e 1 gioco proposto dai vecchi lupi.
        Il canto e gioco dei vecchi lupi sarà scelto tra questi 3 (e 3):
        
        Canti:
        \begin{itemize}
            \item Il falco
            \item Cenerentola
            \item La gioia
        \end{itemize}

        Bans:
        \begin{itemize}
            \item Giovanni e i suoi strumenti
            \item TiIaIa
            \item In un giardino ciucciua ciucciua ! 
        \end{itemize}

        \subsubsection{Materiali}
        \begin{itemize}
            \item Fogli e buste. (Akela)
        \end{itemize}
        \subsubsection{Note}
        \subsubsection{Spiega}
   \end{document}