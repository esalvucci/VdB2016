\documentclass[../main.tex]{subfiles}   
   \begin{document}
   \section{Mercoledi 3/7}
   \subsection{Timetable}
   \begin{itemize}
        \item 7:30 Sveglia
        \item 8:00 - 10:00 Ginnastica, colazione e ispezione
        \item 10:00 - 12:30 Bottega
        \item 12:30 - 15:00 Pranzo, servizi, TL
        \item 15:00 - 16:30 Gioco Cluedo
        \item 16:30 - 17:30 Merenda + Catechesi
        \item 17:30 - 19:30 \textbf{Gioco}
        \item 19:30 - 21:00 Cena
        \item 21:00 - 22:30 Gioco Notturno
        \item 22:30 Nanna!
    \end{itemize}

   \subsection{10:00 - 12:30 Bottega}
   cfr \textbf{Botteghe}

    \subsection{15:00 - 16:30 Gioco Cluedo}
        \subsubsection{Lancio}
        Boromir viene trovato morto, con la testa insanguinata, chi è stato?
        \subsubsection{Gioco}
        I personaggi sono:
        \begin{itemize}
            \item Gollum\\Molto amico di boromir, gli aveva promesso un posto al suo fianco quando sarebbe diventato sovrano della terra di mezzo, grazie al potere dell’anello.
            \item Eowyn\\Dormiva, è stata svegliata da forti rumori. La prima a giungere sulla scena, subiito dopo arrivato galadriel gollum e frodo
            \item Frodo\\Dice che era a guardare le stelle, non si è accorto di nulla fino a che non l’hanno chamato, e getta i suoi sospetti sulle ragazze. 
            \item Galadriel\\ha donato a frodo un pezzo della mappa per trovare l’anello, arrabbiata con lui perché non l’aveva ringraziata né ricompensata. Al momento dell’omicidio stava guardando nel pozzo modificando negativamente il futuro di frodo.
            \item Elrond\\non ha visto nulla, dormiva, ma ha sentito che ieri boromir e una delle ragazze (eowyn o galadriel) litigavano sull’anello
            \item Boromir\\non ricorda come è morto. L’ultimo ricordo è che aveva mangiato del pan di via. 
        \end{itemize}
        \subsubsection{Materiali}
       \subsubsection{Note}
       \subsubsection{Spiega}
       
    \subsection{16:30 - 17:30 Catechesi}   
    \subsection{17:30 - 19:30 Hobbottino}
        \subsubsection{Lancio}
        \subsubsection{Gioco}
        Roverino
        \subsubsection{Materiali}
        \begin{itemize}
            \item pali
            \item roverino
        \end{itemize}
        \subsubsection{Note}
        \subsubsection{Spiega}
        
    \subsection{21:00 - 22:30 Gioco notturno}
    Dopo cena si fa un momento canti breve ma plausibile, si finge di mandarli a letto.
        \subsubsection{Lancio}
        Gandalf frodo e sam fanno chiamata. La battaglia incombe: ma da soli non ce la si può fare, serve l’aiuto di rohan, dobbiamo chiamarlo!
        \subsubsection{Gioco}
        Prima prova: tutti insieme, ogni mistiglia ha cinque parole, bisogna che trovi la parola che le unisce tutte. (ghigliottina di carlo conti) uscirà il colore della mistiglia.\\
        \\
        Rosso nani\\
\\
Pomodoro - Vergogna - Relativo - Bologna - Bandiera 
\\
\\
Uomini blu\\
\\
Oltremare - Notte - Fifa - Maglione - Cielo
\\
\\
Elfi verde\\
\\
Invidia - Prato - Speranza - Elfo - Trifoglio\\
\\
\\
Hobbit giallo\\
\\
Sole - Fiume - Miele - Oro - Crema
\\
Intanto aragorn galadriel eowin legolas saranno nel buio, vicino ad un punto fuoco già preparato, appena tutte le mistiglie avranno indovinato il colore, con un segnale luminoso (accendere una fiaccola?) gandalf li avviserà. Loro devono accendere il fuoco (diavolina, alcool, torce antivento?) e buttarci sopra una polvere che farà diventare il fuoco del colore della mistiglia. A quel punto i ffss corrono al fuoco corrispondente.\\
\\
Completata la prova si girerà. Le prove saranno a punteggio.\\
\\
GIOCO LEGOLAS: cheerleading (mang deve spiegare cosa vuol fare)\\
\\
GIOCO EOWIN: indovinelli che fa gollum ne lo hobbit.\\
\\
GIOCO GALADRIEL: percorso nel buio (Bagheera sa, deve comunicarlo a quelli che non sono elfi immortali)\\
\\
GIOCO ARAGORN: Staffetta (avevo pensato alla mela in due. Ma si fa già la domenica, ergo le alternative sono o un telefono senza fili in movimento oppure la staffetta delle barche (Wontolla sa ma vogliamo sapere anche noi))
        \subsubsection{Materiali}
        \begin{itemize}
            \item per segnale luminoso
            \item per accendere i fuochi
            \item NACL (sale) - Giallo
            \item cloruro di stronzo! - Rosso
            \item cloruro di rame - Blu
            \item solfato di rame - Verde
        \end{itemize}
        \subsubsection{Note}
        \subsubsection{Spiega}
   \end{document}