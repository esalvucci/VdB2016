\documentclass[../main.tex]{subfiles}   
   \begin{document}
    \section{Domenica 31/7}
    \subsection{Timetable}
    \begin{itemize}
        \item 8:30 Ritrovo + Verifica dell'anno + propositi per le vdb
        \item 11:00 Messa
        \item 12:30 - 15:00 Pappa + carico + Pronti... partenza... Via!
        \item 15:00 - 16:30 Arrivo a Limisano con il pullman
        \item 16:30 - 17:30 Strada fino alla casa
        \item 17:30 - 19:30 Gioco regole + lancio servizi + sistemazione zaini nelle camere
        \item 19:30 - 21:00 Cena
        \item 21:00 - 22:30 Fuoco
        \item 22:30 Nanna!
    \end{itemize}

%       Template per la struttura di una singola attività       %    
%    \subsection{Gioco regole}
%       \subsubsection{Lancio}
%        \subsubsection{Gioco}
%        \subsubsection{Materiali}
%        \subsubsection{Note}

    \subsection{17:30 - 19:30 Gioco regole}
       \subsubsection{Lancio}
        \subsubsection{Gioco}
        4 botteghe e 4 gruppi che ruotano
        
        \begin{itemize}
            \item   \textbf{Si fa silenzio quando qualcuno parla}\\
                    Per questa bottega si faranno due giochi: Nel primo caso faremo ascoltare dei versi di animali registrati e modificati con del rumore. I ff/ss dovranno riconoscere gli animali.
Il secondo gioco è simile: i fratellini e le sorelline ascolteranno una canzone giungla registrata. Nella canzone ci saranno parole “intruse” che loro stessi dovranno riconoscere.\\
            \\
            Il vecchio lupo di questa bottega lancierà i servizi di \textbf{Animazione e giornalino} (vedi ~\ref{sec:note})\\
            \\
            \item  \textbf{Luoghi tabu}\\
            Prato fiorito tipo il gioco di Windows '95/'98/2000/XP:
            Dividiamo i fratellini e le sorelline in due sottogruppi (saranno circa 7 o 8 per cui faremo sottogruppi da 3,5 o 4 persone).
            Ci sarà una griglia/scacchiera con dentro dei fogli capovolti. I fogli potranno avere simboli "buoni" oppure una "bomba" (simbolo del tabù).\\Ogni fratellino o sorellina dovrà posizionarsi su una casella e girare il foglio; se nel foglio ci sarà disegnata la bomba allora il fratellino è esploso (anzi, solo eliminato dai...). Se nel disegno ci sarà qualcosa di diverso dalla bomba il fratellino è salvo e potrà scegliere un'altra casella quando sarà di nuovo il suo turno (quindi quando tutti gli altri avranno fatto la loro mossa).\\
            \\
            Le squadre giocheranno contemporaneamente sullo stesso campo (quindi prima componente della squadra 1 e poi quello della squadra 2 e così via a turno).\\
            Vince chi fa esplodere meno bombe oppure la squadra che ha più componenti quando tutte le caselle sono aperte.\\
            \\
            Variante: Una delle figure rappresentate sarà una cucina/cambusa. Chi trova questa figura è salvo (per una volta) alla prima bomba che trova
            
           In questo modo lanceremo il servizio "Aiuto cambu".\\
            \\
            Il vecchio lupo di questa bottega lancierà i servizi \textbf{Aiuto cambu} (vedi ~\ref{sec:note})\\
            \\
        \item
        \textbf{Si mangia quello che c'è}\\
Staffetta con la bacinella d'acqua. I fratellini e le sorelline dovranno portare in salvo i frutti dentro una bacinella senza usare le mani ma solo addentandoli. I frutti che non saranno caduti alla fine della staffetta li mangeranno i fratellini e sorelline da merenda.

Alla fine del gioco si spiegherà la regola specificando bene che:
- Se sai che non ti piace chiedine poco
- Quello che entra nel piatto lo mangi e non rompi
- Se chiedi il bis/tris lo finisci, non vale la storia che non ti va più perché ne hai già mangiato tanto\\
            \\
            Il vecchio lupo di questa bottega lancierà i servizi di \textbf{Catechesi} (vedi ~\ref{sec:note})\\
            \\
        \item
         \textbf{Si tiene in ordine}
Gli diamo dei vestiti, con questi i fratellini e le sorelline dovranno “disegnare” qualcosa disponendo in ordine i vestiti stessi (come faceva Neil in ArtAttack).
Le cose da disegnare saranno:
- Un albero
- Mor il pavone
- La Waingunga e ciò che ci stà attorno
- Scritta “Branco” con ogni lettera di un colore diverso\\
            \\
            Il vecchio lupo di questa bottega lancierà i servizi di \textbf{Pulizia} (vedi ~\ref{sec:note})\\
            \\
        \end{itemize}
        
        \subsubsection{Materiali}
            \begin{itemize}
                \item File audio degli animali  (Akela)
                \item File audio della canzone  (Akela)
                \item Bacinella per l'acqua
                \item frutta (Cambusa)
                \item Cruciverba su San Francesco (Akela)
                \item Immagine con cose che non vanno bene da trovare (gioco servizio pulizia)    (Akela)
            \end{itemize}
        \subsubsection{Note}
        \label{sec:note}
        Questo gioco prevede anche il lancio dei servizi:\\
        Alla fine dell'ultimo turno delle botteghe (quando saranno state fatte tutte e 4 le rotazioni dei gruppi iniziali) il vecchio lupo di ogni bottega darà al gruppo che ha in quel momento un biglietto, indizio per una mini caccia al tesoro.\\
        \\
        Mentre i fratellini cercano il posto i vecchi lupi vanno nel luogo indicato, dove faranno fare al gruppo che arriva un gioco per lanciare uno o più servizi (alcuni li abbiamo accorpati perchè avevamo più servizi che regole).\\
        \\
        \begin{itemize}
            \item Catechesi: Cruciverba su San Francesco
            \item Giornalino: Ogni fratellino
            \item Pulizia (bagni+campo): Immagine con cose fuori posto o in disordine. I fratellini dovranno trovare cosa c'è che non va.
            \item Animazione: I fratellini proporranno 2 giochi e 2 canti che gli piacerebbe fare durante le vdb (Gli Anni non vale!!!)
            \item Aiuto cambu: dire 10 modi per incrostare una padella.
        \end{itemize}
        
        Concluse queste brevi prove si tornerà in cerchio e ogni gruppo condividerà il servizio scoperto.
        
        Finito tutto i gruppi, a turno, sistemeranno gli zaini nelle camere. Chi rimane in cerchio farà ban e giochi
    
    \subsection{21:00 - 22:30 Lancio tema + mistiglie + posta + canzone del campo}
    \subsubsection{Lancio}
    \subsubsection{Gioco}
    \subsubsection{Materiali}
    \subsubsection{Note}

\end{document}