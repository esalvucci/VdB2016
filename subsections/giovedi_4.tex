\documentclass[../main.tex]{subfiles}   
   \begin{document}
   \section{Giovedi 4/7}
   \subsection{Timetable}
   \begin{itemize}
        \item 7:30 Sveglia
        \item 8:00 - 10:00 Ginnastica, colazione e ispezione
        \item 10:00 - 12:30 Bottega
        \item 12:30 - 15:00 Pranzo, servizi, TL
        \item 15:00 - 16:30 Gioco Sfondamento (Battaglia)
        \item 16:30 - 17:30 Merenda + Catechesi
        \item 17:30 - 19:30 \textbf{Gioco}
        \item 19:30 - 21:00 Cena
        \item 21:00 - 22:30 Fuoco
        \item 22:30 Nanna!
    \end{itemize}

   \subsection{10:00 - 12:30 Bottega}
   cfr \textbf{Botteghe}

    \subsection{15:00 - 16:30 Gioco Sfondamenteo (Battaglia)}
        \subsubsection{Lancio}
        \subsubsection{Gioco}
        Ci sono 4 basi: ogni base ha
        10 scalpi.
        Ogni squadra ha un vecchio lupo di riferimento.
        Ogni fratellino o sorellina avrà un biglietto
        con un numero da 1 a 10. Su alcuni biglietti
        si sarà scritto o “Elmo” o “Spada” invece che il numero.
        Ad ogni componente della mistiglia verrà dato un biglietto (del colore della propria mistiglia) con il numero/ruolo\\
        \\
        Obiettivo: al fischio finale del gioco vincerà la squadra che avrà il maggior numero di scalpi.
        Come si prendono gli scalpi? Sfida al tocco: A questo punto i due fratellini/sorelline confronteranno il numero ricevuto. Vince chi ha il numero più basso. Chi vince prende entrambi i biglietti e li porta nella propria base.

        Se un fratellino/sorellina entra nella base senza che nessuno lo sfidi allora può prendere uno scalpo.

        “Spada” vince su tutti i numeri, Elmo perde con tutti ma vince con Spada.

        \subsubsection{Materiali}
        \begin{itemize}
            \item Scalpi
            \item Biglietti (Kaa)
        \end{itemize}
       \subsubsection{Note}
       \subsubsection{Spiega}
       
    \subsection{16:30 - 17:30 Catechesi}   
    \subsection{17:30 - 19:30 Rugby nanetto}
        \subsubsection{Lancio}
        \subsubsection{Gioco}
        Rugby nanetto 
        \subsubsection{Materiali}
        \subsubsection{Note}
        \subsubsection{Spiega}
        
    \subsection{21:00 - 22:30 Fuoco}
        \subsubsection{Lancio}
        4 vecchi personaggi del SdA racconteranno a turno una storia che riguardi i 4 elementi (aria, terra, fuoco, acqua).    
        Una leggenda per ogni popolo (umani, nani, elfi ecc..)

        Mentre il narratore racconta i fratellini e sorelline fanno i quadri fissi della storia raccontata.

        \subsubsection{Gioco}
        Gioco in base alla caratteristica del popolo
        \begin{itemize}
            \item Aria: Mickey Mouse (Raksha)
            \item Fuoco: Danza del fuoco di Chil (F) e Raksha 
            \item Acqua: Vascello di Raksha và! 	(dedicato a borrach!)
            \item Terra: Il ban bellissimo di Raksha
        \end{itemize}
        Per concludere canto: (Ooh) Freedom (canto in onore della fine della battaglia)
        
        \subsubsection{Materiali}
        Storie
        \subsubsection{Note}
        Testo del "ban bellissimo di Raksha"\\
        
        Guarda questo sole come picchia forte (si indica in alto e si fa segno con la mano aperta del sole che picchia)\\
        a pensarci bene il mi faccio un blues (Si fa segno di pensare e si muove il bacino a tempo di musica)\\
        \\
        RIT:\\
        \\
        Siccità uoh oh, siccità uoh oh, (Si muovono le braccia in alto in basso a tempo di musica)\\
        siccitaaaaà uohh oh oh,\\
        siccità in blues!\\
        \\
        prendo una pala scavo una fossa (si mima la presa di una pala e si finge di scavare una fossa)
        E mi butto in faccia un po' di terra mossa(con le mani si fa finta di buttarsi della terra in faccia)
        \\
        RIT\\ 
        (o in alternativa Arrammpampam)
        \subsubsection{Spiega}
   \end{document}